\documentclass{article}

\usepackage[top=1in,left=1.5in,right=1.5in,bottom=1.5in]{geometry}

\usepackage{subcaption}

\usepackage{booktabs}

\usepackage{graphicx}
\let\rfb\reflectbox
\graphicspath{ {images} }

\usepackage{cancel}

\usepackage{mathtools}

\usepackage{nicefrac}
\newcommand{\flippedfrac}[2]{\rfb{\nicefrac{\rfb{#2}}{\rfb{#1}}}}



\usepackage{amsthm}
\renewcommand{\qedsymbol}{$\blacksquare$}
\usepackage{amssymb}
\usepackage{amsmath}
\newcommand\numberthis{\addtocounter{equation}{1}\tag{\theequation}}
\DeclareMathOperator*{\argmax}{arg\,max}
\DeclareMathOperator*{\argmin}{arg\,min}

\usepackage{hyperref}
\hypersetup{
    colorlinks = true,
}


\usepackage{titling}
\title{Exercise Set 4 - Reinforcement Learning}
\newcommand{\subtitle}[1]{%
  \posttitle{%
    \par\end{center}
    \begin{center}\large#1\end{center}
    \vskip0.5em}%
}
\makeatother
\subtitle{Control with approximation and policy gradients}
\author{Giulio Starace - 13010840}
\date{\today}

\begin{document}
\maketitle
\section*{Homework: Geometry of linear value-function approximation (Application)}
\begin{enumerate}
	\item hello world
	\item hello world
	\item hello world
	\item hello world
\end{enumerate}
\section*{Homework: Coding Assignment - Deep Q Networks}
\begin{enumerate}
	\item Coding answers have been submitted on codegra under the group ``stalwart cocky sawly".
	\item hello world
\end{enumerate}

\section*{Homework: REINFORCE}
\begin{enumerate}
	\item hello world
	\item hello world
	\item hello world
	\item hello world
	\item hello world
\end{enumerate}


\end{document}

