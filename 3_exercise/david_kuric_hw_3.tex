\documentclass{article}

\usepackage[top=1in,left=1.5in,right=1.5in,bottom=1.5in]{geometry}

\usepackage{subcaption}

\usepackage{booktabs}

\usepackage{graphicx}
\let\rfb\reflectbox
\graphicspath{ {images} }
 
\usepackage{cancel}

\usepackage{mathtools}

\usepackage{nicefrac}
\newcommand{\flippedfrac}[2]{\rfb{\nicefrac{\rfb{#2}}{\rfb{#1}}}}



\usepackage{amsthm}
\renewcommand{\qedsymbol}{$\blacksquare$}
\usepackage{amssymb}
\usepackage{amsmath}
\newcommand\numberthis{\addtocounter{equation}{1}\tag{\theequation}}
\DeclareMathOperator*{\argmax}{arg\,max}
\DeclareMathOperator*{\argmin}{arg\,min}

\usepackage{hyperref}
\hypersetup{
    colorlinks = true,
}


\usepackage{titling}
\title{Exercise Set 3 - Reinforcement Learning}
\newcommand{\subtitle}[1]{%
  \posttitle{%
    \par\end{center}
    \begin{center}\large#1\end{center}
    \vskip0.5em}%
}
\makeatother
\subtitle{Advanced TD methods and approximation}
\author{Giulio Starace - 13010840}
\date{\today}

\begin{document}
\maketitle
\section*{Homework: Coding Assignment - Temporal Difference Learning}
\begin{enumerate}
	\item Coding answers have been submitted on codegra under the group ``stalwart cocky sawly".
	\item Hello World
\end{enumerate}

\section*{Homework: Maximization Bias}
\begin{enumerate}
	\item For the sake of clarity, we label the four outgoing actions from $B$ as $a_1$, $a_2$, $a_3$
	      and $a_4$, from left to right, and say they belong to the action set $A$. For expected
	      SARSA, we use the expected SARSA update rule to determine the state-action values:
	      \begin{align*}
		      Q(S_t, A_t) & \leftarrow Q(S_t, A_t) + \alpha \left[ R_{t+1} + \gamma \mathbb{E}_\pi
		      \left[Q(S_{t+1}, A_{t+1})|S_{t+1}\right]  - Q(S_t, A_t) \right]                            \\
		                  & = Q(S_t, A_t) + \alpha \left[ R_{t+1} + \gamma \sum_{a \in A} \pi(a|S_{t+1})
			      Q(S_{t+1}, a) - Q(S_t, A_t) \right]. \numberthis \label{eq:expected_sarsa_update}
	      \end{align*}
	      Because all actions from $B$ lead to a terminal state, we have that $Q(S_{t+1}, a) = 0$ for
	      all $a \in A$ when $S_t = B$.

	      For $a_1$, on the first relevant sampled episode we have $R_{t+1} = 0$ giving:
	      \begin{align*}
		      Q(B, a_1) & \leftarrow  0.7 + 0.2 \left[ 0 +\cancelto{0}{1 \times 4(0.25 \times 0)}
		      - 0.7 \right]                                                                       \\
		                & = 0.7 + 0.2 \left[- 0.7 \right]                                         \\
		                & = 0.56.\numberthis                                                      \\
	      \end{align*}
	      And on the next relevant sampled episode we get the same reward, giving:
	      \begin{align*}
		      Q(B, a_1) & \leftarrow 0.56 + 0.2 \left[0 + \cancelto{0}{1 \times 4(0.25 \times 0)} - 0.56
		      \right]                                                                                    \\
		                & = 0.56 + 0.2 \left[- 0.56 \right]                                              \\
		                & = 0.448.\numberthis
	      \end{align*}
	      For $a_2$, on the first relevant sampled episode we have $R_{t+1} = 1$, giving:
	      \begin{align*}
		      Q(B, a_2) & \leftarrow 0.7 + 0.2 \left[ 1 + \cancelto{0}{1 \times 4(0.25 \times 0)} - 0.7
		      \right]                                                                                   \\
		                & = 0.7 + 0.2 \left[0.3 \right]                                                 \\
		                & = 0.76. \numberthis
	      \end{align*}
	      And on the next relevant sampled episode, we get the same reward, giving:
	      \begin{align*}
		      Q(B, a_2) & \leftarrow 0.76 + 0.2 \left[ 1 + \cancelto{0}{1 \times 4(0.25 \times 0)}
		      - 0.76 \right]                                                                       \\
		                & = 0.76 + 0.2 \left[0.24 \right]                                          \\
		                & = 0.808. \numberthis
	      \end{align*}
	      For $a_3$, on the first relevant sampled episode we have $R_{t+1} = 1$, which we know from
	      the first update to $a_2$ gives us
	      \begin{equation}
		      Q(B, a_3) \leftarrow 0.76.
	      \end{equation}
	      On the next relevant sampled episode, we have $R_{t+1} = 0$, giving:
	      \begin{align*}
		      Q(B, a_2) & \leftarrow 0.76 + 0.2 \left[ 0 + \cancelto{0}{1 \times 4(0.25 \times 0)}
		      - 0.76 \right]                                                                       \\
		                & = 0.76 + 0.2 \left[-0.76 \right]                                         \\
		                & = 0.608. \numberthis
	      \end{align*}
	      For $a_4$, on the first relevant sampled episode we have $R_{t+1} = 0$, which we know from
	      the first update to $a_1$ gives us
	      \begin{equation}
		      Q(B, a_4) \leftarrow 0.56.
	      \end{equation}
	      On the next relevant sampled episode, we have $R_{t+1} = 1$, giving:
	      \begin{align*}
		      Q(B, a_1) & \leftarrow 0.56 + 0.2 \left[1 + \cancelto{0}{1 \times 4(0.25 \times 0)} - 0.56
		      \right]                                                                                    \\
		                & = 0.56 + 0.2 \left[0.44 \right]                                                \\
		                & = 0.648.\numberthis
	      \end{align*}
	      For Q-learning, we use the Q-learning update rule to determine the state-action values:
	      \begin{equation}
		      Q(S_t, A_t) \leftarrow Q(S_t, A_t) + \alpha \left[ R_{t+1} + \gamma \max_{a \in A}
			      Q(S_{t+1}, a) - Q(S_t, A_t) \right].\label{eq:q_learning_update}
	      \end{equation}
	      Note once again that since when $S_t = B$, $S_{t+1}$ is always a terminal state, then like
	      before $Q(S_{t+1}, a) = 0$ for all $a \in A$. Therefore, in this case, equation
	      (\ref{eq:q_learning_update}) reduces like equation (\ref{eq:expected_sarsa_update}) to
	      \begin{equation}
		      Q(S_t, A_t) \leftarrow Q(S_t, A_t) + \alpha \left[ R_{t+1} - Q(S_t, A_t)
			      \right]. \label{eq:terminal_reduction}
	      \end{equation}
	      Therefore, all the state-action values in state $B$ are the same in Q-learning as for
	      expected SARSA. For a clearer summary, refer to Table \ref{tab:b_qvalues}.
	      \begin{table}[ht]
		      \centering
		      \caption{Expected SARSA and Q-learning state-action pair values for the four available
			      actions at state $B$ after sampling two episodes per action.}
		      \label{tab:b_qvalues}
		      \begin{tabular}{@{}rcccc@{}}
			      \toprule
			                              & $Q(B, a_1)$ & $Q(B, a_2)$ & $Q(B, a_3)$ & $Q(B, a_4)$ \\ \midrule
			      \textbf{expected SARSA} & 0.448       & 0.808       & 0.608       & 0.648       \\
			      \textbf{Q-learning}     & 0.448       & 0.808       & 0.608       & 0.648       \\ \bottomrule
		      \end{tabular}
	      \end{table}
	\item To determine what the new $Q(A, L)$ value is when executing $L$ in $A$ after the 10
	      episodes, assuming that $Q(A, L)$ is still at 0.7, we use the same update rules as stated
	      before, i.e. equation (\ref{eq:expected_sarsa_update}) for expected SARSA and equation
	      (\ref{eq:q_learning_update}) for Q-learning. Since taking $L$ at $A$ leads to a terminal
	      state, equations (\ref{eq:expected_sarsa_update}) and (\ref{eq:q_learning_update}) once
	      again reduce to equation (\ref{eq:terminal_reduction}). For both expected SARSA and
	      Q-learning we therefore have:
	      \begin{align*}
		      Q(A, L) & \leftarrow 0.7 + 0.2 \left[ 0.7 - 0.7 \right] \\
		              & = 0.7 + 0.2 \left[ 0 \right]                  \\
		              & = 0.7. \numberthis
	      \end{align*}
	      We apply the same process to determine what the new $Q(A, R)$ value is when executing $R$ in
	      $A$ after the 10 episodes, assuming that $Q(A, R)$ is still at 0.7. However, the reduction
	      to equation (\ref{eq:terminal_reduction}) is not possible in this case, since $R$ from $A$
	      does not transition to a terminal state. With expected SARSA we have
	      \begin{align*}
		      Q(A, R) & \leftarrow 0.7 + 0.2 \left[ 0 + 0.25\left(0.448 + 0.808 + 0.608 + 0.648\right)
		      - 0.7\right]                                                                             \\
		              & = 0.6856. \numberthis
	      \end{align*}
	      With Q-learning, we have
	      \begin{align*}
		      Q(A, R) & \leftarrow 0.7 + 0.2 \left[ 0 + 0.808 - 0.7 \right] \\
		              & = 0.7216. \numberthis
	      \end{align*}
	      For a clearer summary, please refer to Table \ref{tab:a_lr_qvalues}.
	      \begin{table}[ht]
		      \centering
		      \caption{Expected SARSA and Q-learning state-action pair values at $A$ when executing $R$
			      and $L$ from $A$ after the 10 sampled episodes.}
		      \label{tab:a_lr_qvalues}
		      \begin{tabular}{@{}ccc@{}}
			      \toprule
			                & \textbf{Expected SARSA} & \textbf{Q-learning} \\ \midrule
			      $Q(A, L)$ & 0.7                     & 0.7                 \\
			      $Q(A, R)$ & 0.6856                  & 0.7216              \\ \bottomrule
		      \end{tabular}
	      \end{table}
	\item Assuming convergence to optimality for both Q-learning and Expected SARSA, we can obtain the
	      true state-action values by utilising the Bellman optimality equation:
	      \begin{equation}
		      q_*(s, a) = \sum_{s', r} p(s', r \mid s, a) \left[ r + \gamma \max_{a'} q_*(s', a') \right].
	      \end{equation}
	      The results of applying this equation to our MDP are summarised in Table
	      \ref{tab:true_qvalues}.
	      \begin{table}[ht]
		      \centering
		      \caption{True state-action values after Expected SARSA and Q-learning convergence.}
		      \label{tab:true_qvalues}
		      \begin{tabular}{@{}cccccc@{}}
			      \toprule
			      \textbf{$Q_*(A, L)$} & \textbf{$Q_*(A, R)$} & $Q_*(B, a_1)$ & $Q_*(B, a_2)$ & $Q_*(B, a_3)$ & $Q_*(B, a_4)$ \\ \midrule
			      0.7                  & 0.5                  & 0.5           & 0.5           & 0.5           & 0.5           \\ \bottomrule
		      \end{tabular}
	      \end{table}
	\item Maximization bias can be observed in all estimated state-action values reported in Tables
	      \ref{tab:b_qvalues} and \ref{tab:a_lr_qvalues}, except for $Q(A, L)$ and $Q(B, a_1)$. We
	      observe maximization bias here as the estimated values are higher than the true values
	      reported in Table \ref{tab:true_qvalues}, i.e. they are positively biased. Both Q-learning
	      and expected SARSA are affected by this bias, as both algorithms rely on a greedy (target)
	      policy which requires the use of a maximum operator. When coupled with stochastic
	      transitions and rewards, such as in our MDP from state $B$, this generally leads to
	      a positive bias in the estimated state-action values.
	\item Double Q-learning circumvents the issue of maximization bias in Q-learning by using two
	      independent estimates, $Q_1$ and $Q_2$, of the true value function $q$. This is unlike
	      vanilla Q-learning where we use a single estimate $Q$. The two estimates afford us the
	      possibility of using one estimate for determining the greedy action $A^* = \argmax_aQ_1(a)$
	      and the other for estimating its value $Q_2(A^*) = Q_2(\argmax_aQ_1(a))$. The latter estimate
	      is then unbiased: $\mathbb{E}\left[Q_2(A^*)\right] = q(A^*)$. We can then repeat the process
	      with $Q_1$ and $Q_2$ swapped to obtain another unbiased estimate. More specifically, in
	      double Q-learning for any given timestep $t$ we would, with probability 0.5, use the
	      following update rule:
	      \begin{equation}
		      Q_1(S_t, A_t) \leftarrow Q_1(A_t, A_t) + \alpha \left[ R_t + \gamma Q_2(S_{t+1},
			      \argmax_{a}Q_1(S_{t+1}, a)) - Q_1(S_t, A_t) \right],
	      \end{equation}
	      and flip the roles of $Q_1$ and $Q_2$ otherwise. For a more concrete example, consider the
	      estimate of $q(A, R)$, whose true value is 0.7 but we have estimated to be 0.7216 in
	      vanilla Q-learning as shown in Table \ref{tab:a_lr_qvalues}. Under Double Q-learning, still
	      assuming an initialization of 0.7, our estimate would now look like:
	      \begin{align*}
		      Q_1(a, r) & \leftarrow 0.7 + 0.2 \left[ 0 + Q_2(b, a_2) - 0.7\right] \\
		                & = 0.7 + 0.2 \left[ 0.7 - 0.7\right]                      \\
		                & = 0.7. \numberthis
	      \end{align*}
	      We can repeat this process with $Q_1$ and $Q_2$ swapped to obtain another unbiased estimate:
	      \begin{align*}
		      Q_2(A, R) & \leftarrow 0.7 + 0.2 \left[ 0 + Q_1(B, a_2) - 0.7\right] \\
		                & = 0.7 + 0.2 \left[ 0.7 - 0.7\right]                      \\
		                & = 0.7. \numberthis
	      \end{align*}
	      We see that our estimate has now been reduced from 0.7216 to 0.7, the true value, thus
	      circumventing the maximization bias issue.
\end{enumerate}

\section*{Homework: Gradient Descent Methods}
\begin{enumerate}
	\item The true value of a state $v_\pi(S_t)$ is defined to be the expected return at that state,
	      i.e.
	      \begin{equation}
		      v_\pi(S_t) = \mathbb{E}\left[ G_t \mid S_t = s \right].
	      \end{equation}
	      Since the Monte Carlo target \textit{is} the return $G_t$, estimated using direct samples of
	      the true value, it has the same expected value as $v_\pi(S_t)$ and is therefore an unbiased
	      estimate of the true value of a state by definition.
	\item The temporal difference (TD) error, $\delta_t$ is given by
	      \begin{equation}
		      \delta_t ~\dot{=}~ R_{t+1} + \gamma V(S_{t+1}) - V(S_t).\label{eq:td_err}
	      \end{equation}
	      A weight update that minimizes the mean squared temporal difference error
	      $\overline{\delta_t}$ can be derived by considering the definition of mean square
	      \textit{value} error $\overline{\text{VE}}$:
	      \begin{equation}
		      \overline{\text{VE}} ~ \dot{=} ~ \sum_{s \in \mathcal{S}} \mu(s) \left[v_\pi(s)
			      - \hat{v}(s, \mathbf{w})\right]^2,
	      \end{equation}
	      where $\hat{v}$ is our approximation and $v_\pi$ is the true value. Here we have the squared
	      error term in the brackets, weighted by ``how much we care about it'', $\mu(s)$ in a sum. We
	      can replace the error term in the brackets with the TD error given in equation
	      (\ref{eq:td_err}), obtaining
	      \begin{equation}
		      \overline{\delta_t} = \sum_{s_t, s_{t+1} \in \mathcal{S}} \mu(s) \left[R_{t+1} + \gamma
			      v_\pi(s_{t+1}) - \hat{v}(s_t, \mathbf{w}))\right]^2,
	      \end{equation}
	      where we are now summing over all possible state transitions $(s_t, s_{t+1})$.  We can then
	      find the weight update rule that minimizes the mean squared TD error by taking the gradient
	      with respect to $\mathbf{w}$:
	      \begin{align*}
		      \mathbf{w}_{t+1} & \leftarrow \mathbf{w}_t - \frac{1}{2} \alpha \nabla\left[R_{t+1}
		      + \gamma v_\pi(s_{t+1}) - \hat{v}(s_t, \mathbf{w}))\right]^2                        \\
		                       & = \mathbf{w}_t + \alpha \left[R_{t+1} + \gamma v_\pi(s_{t+1})
		      - \hat{v}(s_t, \mathbf{w}))\right]^2 \nabla \hat{v}(s_t, \mathbf{w})                \\
		                       & = \mathbf{w}_t + \alpha \delta_t \nabla \hat{v}(s_t,
		      \mathbf{w}). \numberthis
	      \end{align*}
	\item hello world
\end{enumerate}


\end{document}

